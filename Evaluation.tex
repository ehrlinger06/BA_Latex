\chapter{Evaluation}
kapitelstarttext...
\section{Simulationsaufbau}
Zunächst wird auf den Aufbau des verwendeten Simulators eingegangen, sowie auf die ,in der Simulationen verwendeten, Daten.
\subsection{Simulationsframework und Simulatoren}
Für die Durchführung der Simulation der in Kapitel 3 vorgestellten Methodiken wird die co-Simulations Software Mosaik verwendet. Der Mosiak Simulator wurde speziell für Simulation von Smart Grid-Anwendung entwickelt und eignet sich deswegen hervorragend für die im Zuge dieser Arbeit zu erfüllenden Aufgaben. Der Simulator unterstütz zudem die schchnelle Einbing von einem Framework namens PYPOWER. PYPOWER wurde zur Simulation von Stromflüssen entwickelt und ist in der Lage anhand eines Stromnetzes die Ströme der elektrischen Energie zu simulieren. Diese Simulationdes Stromflusses schließt auch den Transportverlust der Niederspannung und den Zusammenhang von Wirk- zu Blindleistung mit ein. Die Art der Integration von PYPOWER in die Mosaik co-Simulations Software, sorgt dafür, dass ein Nutzer PYPOWER nicht außergewöhnlich konfigurieren muss. Ein Stromnetz welches an Mosaik übergeben wird, wird automatisch PYPOWER mitverwenden. Das Stromnetz welches in dieser Arbeit verwendet wird ist ein standardisiertes europäisches Niederspannungsnetz, welches von der IEEE, dem Institut für Elektro- und Elektronikingenieure, für solche Simulation herausgegeben wurde, um Ergebnisse auf einer EU-weiten Ebene vergleichbarer zu machen. Die Methodiken um den Stromfluss im Niederspannungsnetz zu beeinflussen wurden im Kapitel 3 vorgestellt. Dieser Teil des Simulators ist auch der einzige Teil welcher zwischen den verschiedenen Simulationen ausgetauscht wird. Der letzte Baustein welcher für die Simulation benötigt wird, sind die Fahrzeuge, welche geladen werden sollen. Dieser Simulator ist in der Lage anhand der aus dem Netz abgerufen Leistung und der Zeit zu bestimmen, wie sich die Fahrzeugparameter verändern. Die verschiedenen Teile des Simulators benötigen an manchen Stellen Daten von anderen Teilen des Simulators. Die Methodiken, welche in dieser Arbeit entwickelt wurden setzen an den Anschlusspunkten des Niederspannungsnetzes für Privatverbraucher an. Über diese Anschlusspunkte wird unter anderem die Leistung bezogen, welche für das laden der mit diesem Anschlusspunkt verbunden Elektrofahrzeuge verwendet wird. Die aktuell verwendete Methodik erhält vom jeweiligen Anschlusspunkt die aktuell anliegende Spannung. Die mit dem jeweiligen Anschlusspunkt verbunden Elektrofahrzeuge, teilen der Methodik die Ankunftszeit, den Abfahrtszeitunkt, die aktuell möglich Stromstärke beim Laden, sowie den aktuellen Ladestand mit. Das jeweilige Elektrofahrzeug bezieht dann die von der Methodik berechnete mögliche Ladeleistung aus dem Netz, teilt diesem also die Höhe der bezogenen Leistung mit. Die Transformatorlast wird über ein Broadcastsystem vom Transformator aus verteilt, Die Teilnehmeranzahl kann von jeder Methodik, anhand der erhaltenen Nachrichten bestimmt werden. Diese Nachrichten wurden wiederrum von anderen Teilnehmer versandt.\\
\begin{figure}[htb]
\centering
	\includegraphics[width=0.5\textwidth]{img/SimAufbau1.png}
	\caption{Schnittstellen zwischen den Teilen des Simulators}
	\label{Abb_SimAufbau}
\end{figure}

In der Abbildung \ref{Abb_SimAufbau} wird ersichtlich wer welche Daten an wen weitergibt. Die mit der eins markierte Verbindung stellt die Weitergabe von Daten  vom Niederspannungsnetz an die aktuelle verwendete Methodik dar. Dabei wird die aktuell am Anschlusspunkt gemessene Spannung weitergegeben. Die zwei mit zwei und drei markierten Übergänge verdeutlichen den Datenaustausch zwischen der verwendeten Methodik und dem angeschlossen Elektrofahrzeug. Das Elektrofahrzeug sendet die Ankunftszeit, die Abfahrtszeit, die aktuell maximal mögliche Stromstärke und den Ladezustand an die Methodik und erhält im Gegenzug die aktuelle Höhe der aus dem Netz beziehbaren Leistung. Über die Verbindung, welche mit 4 markiert ist, wird dem Netz von Elektrofahrzeug mitgeteilt, welche Leistung aktuell aus dem Netz bezogen wird.
\subsection{Annahmen und verwendete Daten}
Im Zuge dieser Arbeit wird eine nicht unerhebliche Menge an Daten verarbeitet. Die Mehrheit dieser Daten steht in Zusammenhang mit den Elektrofahrzeugen. Die Daten die diese für die Simulation zur Verfügung stellen, wurden vom Deutschem Zentrum für Luft – und Raumfahrt erhoben. Diese Erhebung fand im Zuge einer Mobilitätsstudie statt. Es wurden Teilnehmer für diese Studie aus der Bevölkerung ausgewählt, welche dann ihr Mobilitätsverhalten dokumentiert haben. Diese Mobilitätsverhalten dienen als Grundlage für die Daten der verwendeten Elektrofahrzeuge. Durch die Auswertung dieser Daten wurden Ankunfts- und Abfahrtszeiten, sowie die jeweils gefahrene Wegstrecke bestimmt. Durch die Länge der Wegstrecke kann bestimmt werden wie viel des Ladezustandes auf dieser Strecke verbraucht wird und somit auch der Ladestand bei Ankunft an der Ladestation.\\
Es werden für die Simulationen auch einige Annahmen getroffen. Der Kapazität der Batterie, in welcher die elektrische Energie der Elektrofahrzeuge gespeichert wird, wird auf 36253,11 Wh festgelegt. Dieser Wert durch Auswertung zweier Statistiken ermittelt, welche die meistzugelassen Elektrofahrzeuge in Deutschland beinhalteten und die Batteriekapazität pro Modell. Durch die Verwendung dieser Statistiken wurde eine durchschnittliche Batteriekapazität der Elektrofahrzeuge in Deutschland ermittelt. Dieser Mittelwert stellt einen Kompromiss zu einer individuellen Batteriekapazität dar, stellt allerdings auch sicher vergleichbare Ergebnisse zu erhalten. Die Norm DIN EN 50150 setzt die Normspannung im Deutschem Stromnetz auf 230 Volt, daher wird dieser auch in dieser Arbeit verwendet. Die maximale Trafolast wurde durch eine Simulation ohne Aktivität von Elektrofahrzeugen ermittelt. Die, in dieser Simulation gemessene, Transformatorlast wurde verdoppelt. Diese Verdopplung soll die Belastung des Transformators so gering wie möglich halten, den Ladevorgängen aber dennoch die Möglichkeit bieten auch Leistung abzurufen. Eine weitere Annahme ist die maximale Anzahl an Elektrofahrzeugen, hierbei wurde davon ausgegangen, das ein jeder Haushalt im Niederspannungsnetz über exakt zwei Elektrofahrzeuge verfügt, welche mithilfe eine 22kW Ladegeräts geladen werden. Diese Annahme geht aus Statistiken eines Bundesamtes zurück, welche aussagt, das in ländlich geprägten Gebieten vermehrt größere Haushaltemit zwei oder mehr Personen auftreten. Aus Gründer der Normalisierung zwischen den Anschlusspunkten ans Niederspannungsnetz, wird daher immer von zwei Fahrzeugen je Anschlusspunkt ausgegangen. bei dem in dieser Arbeit verwendtem Netz dem IEEE906 gibt es 55 Anschlüsse für Haushalte, folglich wird von 110 Elektrofahrzeugen ausgegangen.

\section{Simulationsergebnisse}
\subsection{VDE alleine}
\subsection{SA-Part alleine}
\subsection{SA-waitingTime alleine}
\subsection{SA-Part-trafo alleine}
\subsection{SA-waitingTime-trafo alleine}
\section{Analyse und Auswertung}
\subsection{SA-part mit SA-waitingTime}
\subsection{SA-part-trafo mit SA-waitingTime-trafo}
\subsection{VDE mit (SA-part, SA-waitingTime)}
\subsection{(SA-part, SA-waitingTime) mit (SA-part-trafo, SA-waitingTime-trafo)}
