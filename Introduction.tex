\chapter{Einleitung}

Die Zahl der Elektrofahrzeuge in Deutschland nimmt immer weiter zu. Während es 2018 noch 98280 Fahrzeuge (Plug-In Hybride und Elektrofahrzeuge) waren es bereits 2019 66997 Plug-In Fahrzeuge und 83175 Elektrofahrzeuge, also 51892 mehr Fahrzeuge \cite{intro_stat_1}. Sowohl Plug-In Hybride als auch Elektrofahrzeuge können über ein Ladegerät mit elektrischem Strom versorgt werden. Laut einer Statistik über die bevorzugten Ladeorte für Elektrofahrzeuge ist das Zuhause des Fahrzeughalters der beliebteste Ort zum Aufladen des Fahrzeuges \cite{intro_stat_2}. Haushalte in Deutschland sind prinzipiell mit dem Niederspannungsnetz verbunden. \\
Das deutsche Niederspannungsnetz arbeitet gemäß DIN EN 50160 mit Wechselstrom bei 230 Volt Normspannung und einer Frequenz von 50 Hz. Wenn nun aber eine große Last auf ein Niederspannungsnetz wirkt, sinkt die Spannung im Netz ab. Sinkt die Spannung zu weit ab, wird die Leistung der betroffenen Geräte zurückgefahren, dies kann bedeuten, dass Geräte nur noch wenig bis keine Leistung mehr liefern können. Im Falle der bereits erwähnten Ladegeräte würde dieses Zurückfahren der Leistung bedeuten, dass die Länge des Ladevorgangs vergrößert wird, wodurch das Fahrzeug erst später wieder zur vollen Verfügung steht. Nun stellen aber gerade die zunehmenden Ladevorgänge der wachsenden Zahl von Elektrofahrzeugen in Deutschland die betroffenen Niederspannungsnetze vor eine große Herausforderung. Die Herausforderung liegt in der Leistung, die jeder einzelne Ladevorgang benötigt. Die Summe dieser Vorgänge, welche auf das Niederspannungsnetz wirken, können nämlich ein Absinken der Spannung zur Folge haben. Dieses Absinken der Spannung tritt vor allem dann auf, wenn ohnehin schon viele Verbraucher Leistung beziehen. Bei einem hohen Leistungsbezug ohne dem laden vom Elektrofahrzeugen, sorgt der zusätzliche bedarf dafür zu einem noch weiterem Absinken der Spannung. Diese Absinken führt zu einer schlechteren Erfahrung bei der Verwendung des Niederspannungsnetzes, nicht nur beim laden von Elektrofahrzeugen, sondern auch bei der herkömmlichen Verwendung. Die Spannung sinkt nun nicht nur bei einem Teilnehmer, sondern sinkt auch für alle andern Teilnehmer mit ab. Somit kann das laden einiger Teilnehmer andere Teilnehmer vom Laden abhalten. Dies kann dazu führen, dass dies am laden gehinderten Teilnehmer nicht ausreichend laden können und so unfair behandelt werden und eine schlechtere Erfahrung als nötig machen. Um dem Absinken bei steigender Last entgegen zu wirken, sehen sich Netzbetreiber, wie etwa E.ON, gezwungen in ihre Netze zu investieren, um den zukünftigen Belastungen besser standzuhalten. \\
Der Netzbetreiber E.ON hat in einer Pressemitteilung \cite{eon_presse} bekanntgegeben in den nächsten 25 Jahren, also bis zum Jahre 2045, rund 2,5 Milliarden Euro in seine Netze investieren zu wollen. Im Netzgebiet von E.ON gibt es laut ihrer Aussage aktuell etwa 6,5 Millionen konventionelle Pkw, im Jahre 2045 will E.ON in der Lage sein all diese Pkw mit den dann ausgebauten Netzen mit elektrischer Energie zu versorgen. Angesichts dieser Menge an Fahrzeugen ergibt sich auch das Problem der Bereitstellung der möglicherweise abgerufenen Leistung. Das Kohlekraftwerk Neurath in Nordrhein-Westfalen gehört zu den größten Kohlekraftwerken Deutschlands, es besitzt eine Leistung von insgesamt 4400MW (\cite{power}). Bei einer angenommenen Ladeleistung von 22kW pro Elektrofahrzeug wäre da Kraftwerk Neurath nur in der Lage 200000 der 6,5 Millionen Fahrzeuge mit Energie zu versorgen. Alle 6,5 Millionen Fahrzeuge wären unter Verwendung eins 22kW Ladegeräts theoretisch in der Lage insgesamt 143GW an Energie zu beziehen, dies entspricht etwa 2,5-mal der durchschnittlichen bezogen Leistung im deutschen Stromnetz in 2019 (\cite{fraunhofer}). Diese Zahlen machen deutlich, das der hohe potenzielle Leistungsbezug von Elektrofahrzeugen es nötig macht diesen in irgendeiner Weise zu limitieren. Eine solche Limitierung könnte etwa direkt im Niederspannungsnetz erfolgen, durch den dort jeweils vorhanden Transformator. Verteilt man die von der E.ON geschätzte Investitionssumme von 2,5 Milliarden Euro auf die 6,5 Millionen Fahrzeuge, ergeben sich pro Fahrzeug etwa 400 Euro. Der Netzbetreiber schätzt allerdings, dass sich diese Zahl noch senken lässt, etwa durch den Einsatz intelligenter Steuerungen. \\
Ein Ansatz zu einer solchen intelligenten Steuerung wird in dieser Arbeit formuliert. Es wird untersucht, ob die Adaption von Eigenschaften des Aloha Protokolls auf das Niederspannungsnetz dazu beitragen kann, das Laden von Elektroautos und die damit einhergehende Last auf Stromnetz besser zu verteilen. Ein besonderes Augenmerk liegt auf der Verteilung der Leistung zwischen den einzelnen Teilnehmern um eine möglichst faire Verteilung der jeweils zur Verfügung stehenden Last zu erreichen. Zu nächsten werden Ansätze untersucht welche sich mit der Spannungsqualität und der Verteilung zwischen den Teilnehmer und der damit einhergehenden Fairness beschäftigen. Ebenso werden Ansätze betrachtet um die Lasten im Niederspannungsnetz besser zu kontrollieren.