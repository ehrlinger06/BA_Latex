
\chapter{Zusammenfassung Neu} 
The problem of a voltage drop in a low voltage network, after the connected peers drew a too high load, motivates an investigation. The power grid scenario with too many users using the grid, for example, to charge electric cars, can be seen as a multiple access problem known from computer networking. An example of a multiple access protocol
 would be the ALOHA protocol. The aloha protocol was invented in 1970 for coordinating and arbitrating access to a shared communication network, this means it falls into the class of multiple access protocols. There are two forms of the ALOHA protocol, that will be looked at here, pure and slotted ALOHA. Through the usage of this protocol, the aim is to improve the handling of high loads on a low voltage sub-network, which can be created by charging multiple electric cars at once. \\
The evaluation of the methods mentioned beforehand will be done using the co-simulation framework mosaik, which couples a power flow solver(PyPower) and our implementation of the ALOHA algorithm. This co-simulator needs predefined networks for a functioning operation, the ones used here are a low voltage grid from a small town in Bavaria and the IEEE906, to representative European network. The results received after the simulation will show if the ALOHA protocol helps by handling these high loads. The voltage levels over time, the delivered power and the overall usage factor of the network will reveal the differences that can be achieved, when using the ALOHA protocol, both in the pure and in the slotted version, over the charge as you come principle.\\
Another set of results can be received when altering the networks and/or the settings attached to it. A first thing could be modifying the voltage boundary at certain places inside the subnetwork, to avoid a too narrow gap between the start value, without an actual load on the subnetwork, and the lower border, which makes devices cut down on power usage. Another possibility is, to alter the network itself, by adding additional paths to the network, so that a radial network turns into a meshed network, where there can be more than one way from the source to a destination.
