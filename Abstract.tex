\thispagestyle{plain}

\section*{Abstract}
Das Problem eines Spannungsabfalls in einem Niederspannungsnetz, nachdem die angeschlossenen Teilnehmer eine zu hohe Last gezogen hat, motiviert eine Untersuchung. Das Stromnetz-Szenario mit zu vielen Benutzer, die das Stromnetz zum Beispiel zum Aufladen von Elektroautos nutzen, können als ein, aus der Netzwerktechnik bekanntes Problem, des multiple access angesehen werden. Ein Beispiel für einen multiple access Protokoll ist das ALOHA-Protokoll. Das Aloha-Protokoll wurde erfunden in 1970 für die Koordinierung und Vermittlung des Zugangs zu einem gemeinsamen Kommunikationsnetz, dies bedeutet, dass es in die Klasse der Mehrfachzugriffsprotokolle fällt. Das Aloha Netzwerkprotokoll arbeitet dezentral am jeweiligen Teilnehmer ohne zentrale Schnittstelle.Es gibt zwei Formen des ALOHA-Protokolls, pure und slotted ALOHA. Durch die Verwendung dieses Protokolls soll die Handhabung von hohen Lasten verbessert werden, welche auf einem Niederspannungsnetz beim Aufladen mehrerer Elektrofahrzeuge wirken. Das ALOHA Protokoll arbeitet mit Wartezeiten, nachdem eine Kollision aufgetreten ist. Es wurden zwei Varianten zur Bestimmung einer solchen Wartezeit erarbeitet, die erste Variante  verwendet lediglich die aktuelle Anzahl von ladebereiten Teilnehmern, die zweite Variante verwendet zusätzlich fahrzeugspezifische Parameter, wie Abfahrtszeit und Ladestand der Batterie. Neben einem Spannungskontroller wurde auch ein Transformatorkontroller entwickelt, welcher auf die vom Transformatorbezogene Leistung reagiert.\\
Die Bewertung der zuvor genannten Methoden erfolgt mithilfe des co-Simulationsframeworks Mosaiks, das einen Stromflusssimulator (PyPower) und unsere Implementierung des ALOHA-Algorithmus. Dieser Co-Simulator benötigt für einen funktionierenden Betrieb ein vordefiniertes Netz, hier wurde das IEEE906, ein repräsentatives europäischen Niederspannungsnetz, verwendet. Die nach der Simulation erhaltenen Ergebnisse zeigen, dass im vorliegenden Szenario die Verwendung beider ALOHA Varianten Vorteile bezüglich dem herkömmlichen Spannungsregelung bieten. Nicht nur wird die Spannungsqualität verbessert und der Leistungsbezug harmonisiert, sondern auch die Fairness zwischen den Teilnehmer wird gehalten und teils auch verbessert. Die Verbesserungen am Spannungscontroller ermöglichen bereits eine Verbesserung der Spannungswerte, wenn allerdings der Transformatorkontroller ebenfalls verwendet wird, wird auch die Lastverteilung auf das Niederspannungsnetz verbessert. Beide Verbesserungen an sich sind für sich bereits erstrebenswert, das beste Ergebnis zeigte sich aber als beider Controller gemeinsam verwendet wurden. Es zeigt sich auch, dass die Verwendung von Teilnehmerzahl und Fahrzeugparametern zur Bestimmung der Wartezeiten dazu geeignet ist längere Wartezeiten zu bestimmen und somit die Ladevorgänge der einzelnen Teilnehmer besser zu verteilen.
