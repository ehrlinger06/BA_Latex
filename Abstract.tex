\thispagestyle{plain}

\section*{Abstract}
Deutsch:\\
Das Interesse an umweltfreundlicher Mobilität in Deutschland wächst, die Politik unterstützt dieses Interesse etwa durch die Förderung der Elektromobilität. Dieses Förderprogramm führt zu einer schnell steigenden Zahl an Elektrofahrzeugen. Die steigende Anzahl an Elektrofahrzeugen stellt, durch die damit auch steigende Menge an Ladevorgänge das Stromnetz vor eine wachsende Herausforderung. Ein erhöhter Leistungsbezug, etwa durch das Laden von Elektrofahrzeugen, führt im Niederspannungsnetz zu Spannungsabfällen. Da solche Spannungsabfälle nur lokal auftreten und so auch nur einzelne Teilnehmer betreffen, beeinflussen sich auch nur die Qualitätserfahrung mancher Teilnehmer und mindern so die Fairness zwischen den Teilnehmern. Das Stromnetz wird immer dann überlastet, wenn zu viele Teilnehmer Leistung darüber beziehen. Das Stromnetz-Szenario mit zu vielen Benutzer, die das Stromnetz zum Beispiel zum Aufladen von Elektroautos nutzen, können als ein, aus der Netzwerktechnik bekanntes, Problem des multiple access angesehen werden. Das multiple access Problem befasst sich mit dem Umgang einer Menge an Teilnehmern, welche alle mit einem Netzwerk verbunden sind, dieses aber immer nur von einem Teil der Teilnehmer gleichzeitig verwenden werden kann, ohne dabei Probleme zu erzeugen. Das Aloha Netzwerkprotokoll liegt in der Klasse der multiple access Protokolle und arbeitet dezentral ohne eine zentrale Schnittstelle. Das Aloha Protokoll arbeitet mit Wartezeiten, diese sollen die Aktivitäten der Teilnehmer verteilen. Die Wartezeiten werden per Zufall bestimmt und richten sich, in dieser Arbeit, einmal nach der Teilnehmerzahl und bei einer zweiten Variante auch noch nach fahrzeugspezifischen Parametern, wie der Standzeit und dem Ladestand. Der Einsatz des Aloha Protokolls mit den beiden verschieden Varianten der Wartezeitberechnung wird mithilfe des Co-Simulationsframeworks Mosaik simuliert. Für die Simulation wird das IEEE906-Netz als Stromnetz verwendet. Bei der Auswertung der Ergebnisse zeigt sich, dass die beiden Varianten mit Wartezeiten im Vergleich mit einem VDE Spannungskontroller ohne Wartezeiten, bessere Ergebnisse, in Hinsicht auf Transformatorlast und Spannung, ermöglichen. Des Weiteren konnte die Fairness zwischen den Teilnehmer bei den Ladeservices teilweise noch verbessert werden. Diese Ergebnisse führen zu dem Schluss, dass der Einsatz von Wartezeiten beim Laden von Elektrofahrzeugen es ermöglicht, die Fahrzeuge gleichmäßiger und bei höheren Spannungen zu laden. 
\\ \\
English: \\
Interest in environmentally friendly mobility is growing in Germany, and politicians are supporting this interest, for example, by promoting electric mobility. This support program is leading to a rapidly increasing number of electric vehicles. The growing number of electric cars, and the resulting increase in the number of charging processes, pose an increasing challenge to the power grid. Increased power consumption, for example, by charging electric vehicles, leads to voltage drops in the low-voltage grid. Since such voltage drops only occur locally and thus only affect individual participants, they only affect the quality experience of some participants and thus reduce the fairness between the participants. The power grid is always overloaded when too many participants draw power from it. The power grid scenario with too many users using the power grid, for example, to charge electric cars, can be seen as a multiple access problem known from network technology. The multiple access problem deals with handling a set of participants who are all connected to a network, which can only be used by a part of the participants at the same time without causing problems. The Aloha network protocol is in the class of multiple access protocols and works decentralized without a central interface. The Aloha protocol works with waiting times, which distribute the activities of the participants. The waiting times are determined randomly and depend in one case on the number of participants and in a second case also on vehicle-specific parameters, such as standing time and charge level. The use of using the Aloha protocol with the two different variants of waiting time calculation is simulated using the co-simulation framework Mosaic. During the simulation, the power grid used is the IEEE906. The evaluation of the results shows that the two variants with waiting times enable better results in terms of transformer load and voltage compared to a VDE voltage controller without waiting times. Furthermore, the fairness between the participants in the charging services was partly improved. These results lead to the conclusion that the use of waiting times when charging electric vehicles makes it possible to charge the vehicles more evenly and at higher voltages.  
