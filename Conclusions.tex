\chapter{Zusammenfassung/Ausblick}
Im Zuge dieser Arbeit wurden Techniken des Aloha Protokolls auf das Stromnetz adaptiert und so Spannungscontroller mit Fokus auf Fairness entwickelt. Die Ergebnisse der Simulation haben gezeigt, das dies durchausgelungen ist. Alleine die Verwendung eins grundlegendem Aloha Ansatzes fördert die Fairness und die Spannungsqualität. Wenn der Spannungscontroller durch einen Transformatorcontroller erweitert wird, kann die Fairness beibehalten und die Spannungsqualität weiter verbessert werden. Die Verbesserung der Lastverteilung, welche durch den Transformatorcontroller erreicht werden konnte, spricht für seine Verwendung. Das Ziel der Arbeit, die Spannungsqualität und Fairness beim Laden von Elektrofahrzeugen im Niederspannungsnetz zu verbessern, wurde erreicht. Dennoch kann dieses Thema noch weiter ausgearbeitet werden. Bei der Verwendung der Controller werden Zufallszahlen aus zuvor festgelegten Intervallen gezogen. Bei der Bestimmung dieser Zufallszahlen kommt eine statistische Verteilungsfunktion mit Normalverteilung zum Einsatz. Es gibt jedoch auch Verteilungsfunktionen welche nicht gleichverteilt sind, die Verwendung einer solchen Funktion könnte das Ergebnis beeinflussen. Ob dies zu einer weitern Verbesserung führen kann, kann durch weitere Simulationen geklärt werden, welche allerdings im Zuge dieser Arbeit nicht durchgeführt wurden. Im Zuge dieser Arbeit wurden lediglich Simulation durchgeführt, ein Test des entwickelten Controllers in der Realität steht noch aus. Ein solcher Test kann zeigen, ob sich die Verbesserung innerhalb der Simulation auch an realen Bedingungen einstellen und reproduzieren lassen.

