\chapter{Related Work}
Konzepte zur Verbesserung der Fairness beim Laden von Elektrofahrzeugen oder zur Kontrolle von der damit einhergehende Last auf das Stromnetz wurden bereits in anderen Arbeiten erarbeitet. Im Folgenden wird auf einige dieser Konzepte eingegangen, dabei werden die Ansätze in zwei Gruppen eingeteilt. In der ersten Gruppe befinden sich Ansätze die sich damit befassen die Fairness und den Quality of Service Aspekt zu erhöhen. In der zweiten Gruppe wird auf Konzepte eingegangen, welche die Lasten innerhalb des Niederspannungsnetzes limitieren oder verschieben wollen.
\section{Konzepte mit Quality of Service Ansatz}
Die Arbeiten in dieser Kategorie haben das Ziel die Fairness beim Laden von Elektrofahrzeugen zu verbessern. Sie bedienen sich Ideen und Techniken der Netzwerktechnik um dieses Ziel zu erreichen.\\
In Ihrer Arbeit \cite{RW_3_1} legen  Ammar  Alyousef und Hermann de Meer ein Konzept dar, für einen Kontrollmechanismus, welcher dezentral an den Anschlüssen der jeweiligen Ladegeräte ansetzt. Der Kontrollmechanismus überprüft die Einhaltung der Grenzwerte für die Last am Transformator sowie der Spannung an den einzelnen Anschlüssen. Die Zustände der gemessen Werte werden anhand eines Ampelschemas eingeteilt, wobei grün keinen Anlass zu Veränderungen anzeigt, gelb eine leichte Änderung und rot eine drastische Änderung, um die Werte innerhalb der jeweils zulässigen Bereiche zu halten. Sollten die Werte nun einen Anlass vermitteln, welcher eine Änderung notwendig macht, verwendet man hier das Prinzip des TCP-Slow Starts, welches aus der Netzwerktechnik stammt. Der Aufbau des verwendeten Netzes sowie die bereits anliegenden Lasten, ohne die Ladegeräte, wurden aus der Realität übernommen. Es werden insgesamt vier verschiedene Szenarien getestet, kein Ladegerät am Netz, alle Ladegeräte unter Vollast am Netz, sowie zwei verschiedene Smart Charging Ansätze, einmal der in der Arbeit selbst vorgestellte und ein Ansatz, der einen endlichen Automaten verwendet und aus vorhandener Literatur herausgenommen wurde, als Vergleichsinstanz. Das Ergebnis zeigt, das sich die Verwendung der beiden Smart Charging Ansätze, vor allem in der Qualität der zur Verfügung stehenden Elektrizität auszahlt, aber auch bei der Menge der transportierten Energie. Jedoch stellt sich dar, dass der Smart Charging Ansatz, welcher den TCP-Slow Start verwendet, fairer ist bei der Verteilung zwischen den Ladestationen. Die Anordnung der Ladestationen unterscheidet sich allerdings dahingehend von der Verteilung in dieser Arbeit, da hier davon ausgegangen wird, das sich die Ladestationen stärker im Netz verteilen, da mehr von Ihnen angeschlossen wurden. Auch in der hier vorgestellten Arbeit unterscheiden sich die verwenden Parameter von denen in dieser Arbeit verwendeten. In der hier vorgestellten Arbeit orientriert man sich lediglich an den Auslastungsdaten des Stromnetzes, während in dieser Arbeit auch Daten der jeweiligen Elektrofahrzeugs berücksichtigt werden, wie Ladezustand des Akkus, sowie der nächste Abfahrtszeitpunkt. \\
Emin Ucer et al. verfolgen in ihrer Arbeit \cite{RWTCP} das Ziel eine Methodik zu entwickeln, welche die Fairness beim Laden von Elektrofahrzeugen steigern soll. Sie lehnen ihre Bemühung an Prinzipien des TCP Protokolls an.  Im Zuge der Arbeit wurde ein additive increase multipicative decrease (AIMD) Kontroller entwickelt. Dieser Kontroller arbeitet komplett dezentral und setzt am Anschlusspunkt zum Niederspannungsnetz an. Als Motivation für die Entscheidung für das AIMD Prinzip, wird die Ähnlichkeit des Spannungsabfalls und der Abfall der Übertragungsgeschwindigkeit im Internet bei zunehmender Entfernung angegeben. Hierfür wird als Erstes untersucht wie sich die Spannung im Niederspannungsnetz verhält, bei zunehmender Entfernung vom Transformator. Aufgrund der Ergebnisse dieser Untersuchung werden individuelle Schwellenwerte für die einzelnen Messpunkte festgelegt. Diese werden allerdings nicht nur einmal bestimmt, da sie, wie in der Arbeit festgestellt wird, nicht nur von der Entfernung abhängen, sondern auch von der Last, die auf das Netz wirkt. Die Schwellenwerte beziehen sich auf die aktuell am Anschlusspunkt anliegende Spannung und geben an bis zu welchem Wert der Spannung der Leistungsbezug erhöht wird. Wird eine Spannung unterhalb des Schwellenwertes gemessen, wird die bezogene Leistung reduziert. Die Regulierung des Leistungsbezuges erfolgt über eine Änderung der Stromstärke. Ist die aktuell gemessen Spannung höher als der Schwellenwert, wird ein zuvor festgelegter Wert zur bisherigen Stromstärke hinzuaddiert. Ist die gemessene Spannung zu niedrig, wird die Stromstärke mit einem ebenfalls zuvor festgelegten Faktor multipliziert um den Wert der Stromstärke zu verringern. Dieses Vorgehen entspricht einem AIMD Kontroller. Die individuell bestimmten Spannungswerte tragen dazu bei, die Fairness im Netz zu erhöhen, das sie schlechter gestellten Nutzer mehr Spielraum einräumen und dieser so mehr und länger laden können. Wohingegen besser gestellte Nutzer höhere Schwellenwerte verwenden, allerdings durch ihr höheres Niveau an Spannung auch mehr Leistung beziehen können.\\
Der AIMD-Kontroller arbeitet nach dem FDM Prinzip, dem Frequenzy Division Muiltiplexing Prinzip, er teilt das Stromnetz zwischen allen Teilnehmer auf, sodass möglichst viele dieses zeitgleich verwenden können. Im Gegensatz dazu arbeitet das in dieser Arbeit verwendete Aloha Protokoll nach dem TDM Prinzip, dem Time Division Muiltiplexing Prinzip, welche Nutzer entlang der Zeit verteilt. Das Netz welches zur Simulation des Kontrollers verwendet wurde, wurde speziell für diese Arbeit erstellt und ist kein Referenznetz, welches bereits anderweitig verwendet wurde. Die Datensätze für die Elektrofahrzeuge und die Grundlast wurden durch Anwendung von stochastischen Verteilungen bestimmt. Diese Datensätze wurden speziell für diese Arbeit erdacht. Zum Ende der Arbeit merken die Autoren an, das Test mit Daten näher an der Realität noch ausstehen. Die Fairness wird nur über die unterschiedlichen Schwellenwerte der Spannung ausgedrückt. Weitere Parameter von Elektrofahrzeugen werden bei der Bestimmung der Ladeenergie nicht berücksichtigt. Allerdings wird auch diese Tatsache erwähnt und eingeräumt, sowie auf eine potenzielle Weiterarbeit verwiesen. 
\section{Konzepte zur Verbesserung der Netzauslastung}
Die Arbeiten welche in diesem Kapitel vorgestellt werden haben zum Ziel die Lasten, die auf das verwendete Netz wirken zur verringern oder um zu verteilen. Dabei verwenden sie verschieden Ansätze, wirtschaftliche Anreize oder eine dezentrale Veteilung.\\
S. Sangob und S. Sirisumrannukul \cite{RW_1_1} haben in Ihrer Arbeit ebenfalls das Ziel das Spannungslevel auch bei mehreren Ladevorgängen stabil zu halten, und so das Netz bestmöglich zu nutzen. Dieses Ziel versuchen sie über eine Partikelschwarmoptimierung zu erreichen, welche auf allen drei Phasen eins 120V Netzes agiert. Das Ergebnis dieser Optimierung ist eine etwa 15\% höhere Leistungsabgabe des Transformator, welche durch eine Erhöhung des Spannungsniveaus bei gleichbleibender Stromstärke erreicht wurde. Die von Ihnen angestrebte Optimierung greift  am Transformator des Niederspannungsnetzes, sowie den mit dem Transformator verbunden Kondensatoren an, also an anderer Stelle, als die in dieser Arbeit thematisierte Lösung, welche am Hausanschluss bzw. erst am Ladegerät selbst ansetzt. Dieser Unterschied beeinflusst auch, an welchem Punkt des Netzes die Spannungswerte gemessen werden, welche von Ihnen nur am Transformator erfasst werden, während die Werte in dieser Arbeit an allen Anschlusspunkten berücksichtigt werden, was die Übertragungsverluste und die Netztopologie mehr berücksichtigt. Ebenso unterscheiden sich die Zielen zwischen der hier genannten Arbeit und dieser Arbeit, während in der hier genannten Arbeit das Ziel war die Qualität der übertragen Spannung zu erhöhen, ist das Ziel dieser Arbeit die Quality of Service des Ladevorgangs von Elektrofahrzeugen, abhängig von Ladezustand des verbauten Akkus und der für den Ladevorgang verfügbaren Zeit, zu erhöhen.
\\
Einen andern Ansatz verfolgen M. Nour et al. in Ihrer Arbeit \cite{RW_2_1}. In Ihrer Arbeit stellen Sie einen Ansatz vor, indem Stromanbieter einen Zweipreistarif anbieten, ein höherer Preis für Zeiten mit höherer Last und ein zweiter, niedrigerer Preis, bei geringerer Last, bestimmt wird diese Auslastung am Transformator des Stromnetzes. Dieses Tarifsystem macht das laden außerhalb von Lastzeiten wirtschaftlich attraktiver, was dazu führen soll, das Halter von Elektrofahrzeugen diese Zeiten zum Laden nutzen und eben nicht die Zeiten, wo auch ohne Ladevorgänge schon eine hohe Last auf dem Netz liegt. Diese Zeitsteuerung wird in einen Fuzzy-Kontroller integriert, welcher neben dem Preis auch den Ladezustand des Fahrzeugs berücksichtigt und so die mögliche Ladeleistung des Fahrzeuges bestimmt. Jedoch steht anders als in dieser Arbeit nicht der Quality of Service Aspekt, einer möglichst zeitnahen,  dem nächsten Abfahrtszeitpunkt entsprechende Ladung im Vordergrund, sonder eher der wirtschaftliche Aspekt, mit der Verwendung von möglichst günstig zur Verfügung stehender Elektrizität. Durch die unterschiedliche Ziele der Arbeiten werden auch unterschiedliche Daten herangezogen, in der hier vorgestellten Arbeit wird nur der Ladezustand des verbauten Akkus betrachtet, während in dieser Arbeit auch die Zeit welche das Fahrzeug am Ladegerät verbracht hat bzw. noch verbringen wird. Des Weiteren kontrolliert die hier vorgestellte Arbeit die Auslastung des Netzes nur passiv, da die Preise nur fallen, wenn die Auslastung niedrig ist und so die Belastung durch die Ladevorgänge verkraftbar ist.\\
Einen Ansatz mit zentraler Verteilung der elektrischen Energie wird von Yingjie Zhou et al. erarbeitet \cite{RWcentral}. Diese Arbeit verwendet nicht nur das Niederspannungsnetz sondern auch höhere Schichten des Stromnetzes. Durch die Einbindung höherer Schichten kann die Energiemenge bestimmt werden, welche im Niederspannungsnetz tatsächlich zur Verfügung steht. Durch diese Maßnahme wird auch eine Limitierung der Strommenge im Niederspannungsnetz erreicht, da nicht mehr Strom bezogen werden kann als verfügbar ist. Im Vordergrund steht die Maßgabe nur soviel Energie zum Laden zu verwenden wie auch verfügbar ist. Die Menge an verfügbarer Energie wird durch die höheren Ebene des Stromnetzes aber auch durch die sonstige Aktivität im Niederspannungsnetz limitiert. Übersteigt die Nachfrage an Energie das aktuelle Angebot wählt eine zentrale Stelle eine Menge an Fahrzeugen aus, deren Bedarf auch gedeckt werden kann. Die Auswahl erfolgt nicht zufällig, sondern anhand spezieller Techniken. Im Zuge dieser Arbeit werden fünf verschiedene Techniken betrachtet. Es werden Round Robin, First come first serve, First depart first serve, Min-max Energy Requirement und min-Max Delay jeweils einzeln verwendet und verglichen. Neben den verwendeten Techniken unterscheidet sich auch das Ziel welchem beim Laden angestrebt wird. Während in dieser Arbeit das Ziel immer ist ein Fahrzeug auf 100\% Ladezustand zu laden, wird bei der hier vorgestellten die nächste geplante Strecke bedacht und dahingehend der dafür notwendige Ladezustand bestimmt und als zu erreichenden Ladestand festgelegt. Eine Ladung darüber hinaus erfolgt nur wenn Kapazität verfügbar ist.



%-kommunukationprotokolle im EV charging
%fair queueing aus 2013 nimmt auch protokolle
%klimaanlage  in mim/max aufteilung

%tcp slowstart mit ev (amar)
