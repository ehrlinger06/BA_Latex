\chapter{Related Work}
\newpage
\section{Smartgrid Konzepte mit ähnlichem Ziel, aber anderer Umsetzung}
Die Bemühungen das bestehende Stromnetz auf die Zukunft vorzubereiten und dafür mehr Inteligenz in das Netz zu integrieren sind nicht nur auf die in dieser Arbeit beschränkt. \\
S. Sangob und S. Sirisumrannukul haben in Ihrer Arbeit mit dem Titel ''Volt/Var Control with Electric Vehicles Loads in Distribution Network by Partical Swarm Optimization'' \cite{RW_1_1}  ebenfalls das Ziel das Spannungslevel Auch bei mehreren Ladevorgängen stabil zu halten, und so das Netz bestmöglich zu nutzen. Dieses Ziel versuchen Sie über eine Partikelschwarmoptimierung zu erreichen, welche auf allen drei Phasen eins 120V Netzes agiert. Das Ergebnis dieser Optimierung ist eine etwa 15\% Reduktion der Spannungsverluste, was in einer höheren Leistungsabgabe des Transformator resultiert. Die von Ihnen angestrebte Optimierung greift  am Transformator des Niederspannungsnetzes, sowie den mit dem Transformator verbunden Kondensatoren an, also an anderer Stelle, als die in dieser Arbeit thematisierte Lösung, welche am Hausanschluss bzw. erst am Ladegerät selbst ansetzt. Dieser Unterschied beeinflusst auch, an welchem Punkt des Netzes die Spannungswerte gemessen werden, welche von Ihnen nur am Transformator erfasst werden, während die Werte in dieser Arbeit an allen Anschlusspunkten berücksichtigt werden, was die Übertragungsverluste und die Netztopologie mehr berücksichtigt.
\\
Einen Ansatz ähnlicher zu dem Ansatz hier verfolgen M. Nour, S. M. Said, A. Ali und C. Farkas in Ihrer Arbeit mit dem Titel ''Smart Charging of Electric Vehicles According to Electric Price'' \cite{RW_2_1}. In Ihrer Arbeit stellen Sie einen Ansatz vor, indem Stromanbieter einen Zweipreistarif anbieten, ein höherer Preis für Zeiten mit höherer Last und ein zweiter, niedrigerer preis, bei geringerer Last, bestimmt wird diese Auslastung am Transformator des Stromnetzes. Dieses Tarifsystem macht das laden außerhalb von Lastzeiten wirtschaftlich attraktiver, was dazu führen soll, das Halter von Elektrofahrzeugen diese Zeiten zum Laden nutzen und eben nicht die Zeiten, wo auch ohne Ladevorgänge schon eine hohe Last auf dem Netz liegt. Diese Zeitsteuerung wird in einen Fuzzy-Controller integriert, welcher neben dem Preis auch den Ladezustand des Fahrzeugs berücksichtigt und so die mögliche Ladeleistung des Fahrzeuges bestimmt. Jedoch steht anders als in dieser Arbeit nicht der Quality of Service Aspekt, einer möglichst zeitnahen,  dem nächsten Abfahrtszeitpunkt entsprechende Ladung im Vordergrund, sonder eher der wirtschaftliche Aspekt, mit der Verwendung von möglichst günstig zur verfügungstehender Elektrizität. Anders als in dieser Arbeit, welche Fahrzeugdaten sowie Netzauslastungsdaten verwendet, die durch Statistiken und Umfragen erhoben wurden, werden hier teils Daten verwendet, welche in der Arbeit einmalig festgelegt wurden und nur für die Netzauslastung wurden statistisch erhoben Daten herangezogen. 

\section{Smartgrid Konzepte mit ähnlichem Ziel, welche ebefalls anderweitig bekannte Techniken abwandeln}
Anders als die bisher vorgestellten Konzepte, welche zwar ein ähnliches oder das selbe Ziel haben, wie diese Arbeit, aber einen anderen Weg eingeschlagen haben um dieses zu erreichen, gibt es auch Ansätze, welche sich Ideen und Konzepten der Netzwerktechnik bedienen und diese auf das Stromnetz anwenden. \\
In Ihrer Arbeit mit dem Titel ''Design of a TCP-like Smart Charging Controller for Power Quality in Electrical Distribution Systems''\cite{RW_3_1} legen  Ammar  Alyousef und Hermann de Meer ein Konzept dar, für einen Kontrollmechanismus, welcher dezentral an den Anschlüssen der jeweiligen Ladegeräte ansetzt. Der Kontrollmechanismus überprüft die Einhaltung der Grenzwerte für die last am Transformator sowie der Spannung an den einzelnen Anschlüssen. Die Zustände der gemessen Werte werden anhand eines Ampelschemas eingeteilt, wobei grün keinen Anlass zu Veränderungen Anzeigt, gelb eine leichte Änderung und rot eine drastische Änderung, um die Werte innerhalb der jeweils zulässigen Bereiche zuhalten. Sollten die Werte nun einen Anlass vermitteln, welcher eine Änderung notwendig macht, verwendet man hier das Prinzip des TCP-Slow Starts, welches aus der Netzwerktechnik stammt. Der Aufbau des verwendeten Netzes sowie die bereits anliegenden Lasten, ohne die Ladegeräte, wurden aus der Realität übernommen. Es werden insgesamt vier verschiedene Szenarien getestet, kein Ladegerät am Netz, alle Ladegeräte unter Vollast am Netz, sowie zwei verschiedene Smart Charging Ansätze, einmal der in der Arbeit selbst vorgestellte und ein Ansatz, der einen endlichen Automaten verwendet und aus vorhandener Literatur herausgenommen wurde, als Vergleichsinstanz. Das Ergebnis zeigt, das sich die Verwendung der beiden Smart Charging Ansätze, vor allem in der Qualität der zur Verfügung stehenden Elektrizität auszahlt, aber auch bei der Menge der transportierten Energie. jedoch stellt sich dar das der Smart Charging Ansatz, welcher den TCP-Slow Start verwendet, fairer ist bei der Verteilung zwischen den Ladestationen. Die Anordnung der Ladestationen unterscheidet sich allerdings dahingehend von der Verteilung in dieser Arbeit, da hier davon ausgegangen wird, das sich die Ladestationen stärker im Netz verteilen, da mehr von Ihnen angeschlossen wurden.
\\
\\




%-kommunukationprotokolle im EV charging
%fair queueing aus 2013 nimmt auch protokolle
%klimaanlage  in mim/max aufteilung

%tcp slowstart mit ev (amar)
