\chapter{Ein Kapitel des Hauptteils}
\label{chap:body}

\section{Inhalte}
Im Hauptteil werden aufbauend die Vorgehensweise zur Problemlösung und 
Ergebnisse der Arbeit im Detail vorgestellt.
Dazu kann der Hauptteil in verschiedene Abschnitte 
(section) oder sogar in mehrere Kapitel (chapter) unterteilt werden.

Einleitung und Hauptteil sollen eine in sich geschlossene Abhandlung
darstellen. Der Leser der Arbeit soll ohne zusätzliche Literatur in
der Lage sein, die Arbeit im Zusammenhang zu verstehen.
Gutes wissenschaftliches Arbeiten (Referenzen angeben, konkrete Sprache etc.) ist ein Muss!

Grob könnte ein Aufbau so aussehen:
\begin{itemize}
	\item Problembeschreibung
	\item Annahmen: Dieser Abschnitt ist sehr wichtig! Hier beschreibt ihr die Grenzen und Annahmen die ihr trefft, damit euer Ansatz funktioniert
	\item Simulationsaufbau/Problemlösung
	\item Ergebnisse: Schaubilder, Grafiken, Tabellen etc.
	\item Evaluation: Was bedeuten die konkreten Ergebnisse?
	\item Ggf. Diskussion der Ergebnisse: Warum diese Werte und nicht andere? War euer Aufbau valide? Was könnte verbessert werden?
\end{itemize}



\section{Goldene 6 \texorpdfstring{$\times$}{x} C - Regel}
Da Verfassen von wissenschaftlichen Arbeiten ein wichtiger Teil an der Universität ist, folgt bitte den goldenen 6 x C Regeln:


\begin{enumerate}
	\item Clarity (Klarheit)
	\item Correctness (Korrektheit)
	\item Conciseness (Prägnanz)
	\item Consistency (Konsistenz)
	\item Connectedness (Verbindung)
	\item Completeness (Vollständigkeit)
\end{enumerate}

Die 6Cs werden bei allen akademischen Texten wie Papern, Doktorarbeiten, Seminararbeiten, Bachelor-und Masterarbeiten angewendet.
Bitte haltet euch an diese Regeln und übt diese.

\section{Struktur des Dokuments}

Die Arbeit sollte wie folgt strukturiert sein:

\begin{enumerate}
	\item Abstract / Kurzfassung
	\item Einleitung \\
	Hier wird ein Überblick über die Arbeit gegeben und das Projekt/Problem beschrieben
	\item (Verwandte Arbeiten)
	\item Hauptteil (gegliedert in mehrere Kapitel oder Abschnitte)
	\item Verwandte Arbeiten (Related Work)
	\item Zusammenfassung, Bewertung und Ausblick
	\item Referenzen / Literatur
\end{enumerate}
Alle Abbildungen in der Arbeiten müssen einen \textbf{Titel und eine Legende/Achsenbeschriftung} haben!
Alle Abbildungen und Gleichungen müssen nummeriert sein, genauso wie Abschnitte, Unterabschnitte und Kapitel.

Referenzen zu Abbildungen, Gleichungen, Abschnitten etc. müssen explizit geschehen, also mit direkten Verweisen 
anstatt ein unkonkreter Beschreibung.



\section{Einreichen}
Bitte folgt den Informationen auf dieser Seite
\url{http://www.net.fim.uni-passau.de/hints}.

Bitte sprecht vorher mit eurem Betreuer ob ihr alle Anforderungen verstanden habt. Vor der Abgabe muss verifiziert werden, dass alle Kopien unterschrieben sind und die elektronische Fassung aktuell ist.



\subsection{Archivierung}
Für die Archivierung sind alle Dateien der Arbeit (auch der Vorträge)
dem Betreuer zur Verfügung zu stellen.  Weiterhin soll noch ein
\BibTeX-Eintrag der Arbeit erstellt werden (die Felder in eckigen
Klammern sind dabei auszufüllen):
\begin{verbatim}
@MastersThesis{<Nachname des Autors><Jahr>,
type =         {<Bachelor- oder Masterarbeit>},
title =        ,
school =       {Institute of Communication Networks~(LKN),
Munich University of Technology~(TUM)},
author =       {<Nachname des Autors>, <Vorname des Autors>},
annote =       {<Nachname des Betreuers>, <Vorname des Betreuers>},
month =        {<Monat>},
year =         {<Jahr>},
key =          {<Mehrere Suchschlüssel>}
}
\end{verbatim}


\chapter{A Chapter of the main part}
In the main part, the procedure how the problem was solved and 
the obtained results of the work are presented in detail.
For this purpose, the main part can be divided into different sections 
or even into several chapters.

Introduction and main part must be written in a conclusive and coherent way.
The reader of the work should be able to read and understand the thesis without additional literature.
Good scientific work (giving references, concrete language etc.) is a must!


Roughly, a structure can be:
\begin{itemize}
	\item Problem description
	\item Assumptions: This section is very important! Here, the limits and assumptions are described which are used for your approach
	\item Simulationsetup/how to solve the problem
	\item Results: Figures, Tables etc.
	\item Evaluation: What do the results mean?
	\item Discussion of results: Why those values? Was your setup valid? What could be improved?
\end{itemize}




\section{Golden 6 \texorpdfstring{$\times$}{x} C - Rule}

Since your are all getting more and more involved into writing,
please follow strictly the golden rule of 6 x C :

This should be explained to you by your supervisor and is about
how to choose your arguments carefully.

\begin{enumerate}
\item Clarity
\item Correctness
\item Conciseness
\item Consistency
\item Connectedness
\item Completeness
\end{enumerate}

The 6Cs apply to all academic texts alike: Papers, PhD theses, Diploma theses, reports etc.
So please practice yourself and teach the students you are supervising these golden rules
as well.


\section{Document Structure}

The project thesis has to be structured according the following standard:

\begin{enumerate}
\item Abstract
\item Introduction\\
This should include an overview of the thesis / project description
(what is explained in which section/chapter)
\item (Verwandte Arbeiten)
\item Main body of work description, divided in several sections/chapters 
\item Related Work
\item Conclusions and Future Work
\item References
\end{enumerate}

All figures in project descriptions / theses must have a \textbf{title and a legend/description of axes!}
All figures and equations must be numbered, as well as sections (reports),
subsections and chapters (books). Please provide also a short description of the message of the figures/tables etc.

References to any object (figures, equations, sections etc.) must be done
explicitly, that is, with pointers, rather than  by means of loose connections
as in shown above. That should be avoided all together. 

\section{Submittal Form}

Please follow the current information provided online at 
\url{http://www.net.fim.uni-passau.de/hints}.

Please check with our supervisor beforehand to ensure you understand all the requirements. Before hand-in, verify that all copies have been signed and that the electronic version is up-to-date.

\subsection{Archive}
For the archive, all files of the work (also presentations) should be made available for the supervisor.
Furthermore, a \BibTeX-entry of the thesis should be created (fill in the fields in square brackets):
\begin{verbatim}
@MastersThesis{<Last name of the author><year>,
type =         {<Bachelor-/ or Master Thesis>},
title =        ,
school =       {Institute of Communication Networks~(LKN),
Munich University of Technology~(TUM)},
author =       {<last name of author>, <first name of author>},
annote =       {<last name of supervisor>, <first name of supervisor>},
month =        {<Month>},
year =         {<Year>},
key =          {<Several key words>}
}
\end{verbatim}
