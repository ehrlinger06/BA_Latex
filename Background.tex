\chapter{Grundlagen}
\label{chap:grundlagen}

Eine Erläuterung und Erklärung der technischen Grundlagen, der beiden zentralen, verwenden Konzepten, das Aloha Protokoll und das Stromnetz sowie auf die Grundlagen von Elektrofahrzeugen.
\section{Aloha-Protokoll}
Das Aloha Protokoll wurde an der Universität von Hawaii entwickelt. Ursprünglich wurde es dort für Übertragungen zwischen Funkstationen entwickelt, allerdings lässt sich das Protokoll überall dort verwenden, wo unkoordinierte Benutzer mit einem geteilten Medium arbeiten \cite{Back_AlohaPure}. Das Aloha Netzwerkprotokoll definiert wie alle Protokolle Regeln und Formate, welche den Ablauf der Kommunikation bestimmen. In der heutigen Form des Internets bzw der Kommunikation über ein Netzwerk, arbeiten mehrere verschiedene Protokolle, welche sich jeweils mit verschiedenen Schritten befassen, zusammen. Diese Zusammenarbeit lässt sich für die Netzwerkkommunikation über das ISO/OSI Schichtenmodell erläutern. Der Weg der Daten von der vom Nutzer verwendeten Anwendung bis zur eigentlichen Aktivität auf einer Leitung eines Netzwerkes wird in sieben Schritte eingeteilt. Die Reihenfolge dieser Schritte beim Senden von Daten ist genau gegensätzlich zu der Reihenfolge beim Empfangen von Daten. Jede Schicht hat dabei ihre spezielle Aufgabe, was sie von den anderen Schichten abgrenzt. Beim Senden von Daten durchläuft man die Schichten in Folgender Reihenfolge, Anwendungsschicht, Dartstellungsschicht, Sitzungsschicht, Transportschicht, Vermittlungsschicht, Sicherungsschicht und Übertragungsschicht. Das Aloha Protokoll arbeitet auf der Sicherungsschicht, auf dieser Ebene soll ein Protokoll in der Lage sein eine fehlerfreie Übertragung zu ermöglichen und den Zugriff auf das Übertragungsmedium zu regeln. Die Daten, welche mithilfe des Aloha Protokolls versendet werden sollen, werden in Frames eingeteilt. In einem solchen Frame werden die Daten in zwei Bereiche unterteilt. Der erste Teil der Daten wird vom Aloha Protokoll selbst benötigt, für die Weiterleitung der Daten, dies ist auch der Teil eines Frames, welcher vom Aloha Protokoll generiert bzw. verarbeitet wird. Der zweite Teil der Daten enthält den eigentlichen Inhalt des Frames, welcher versendet oder empfangen werden soll. Der zweite Teil enthält Daten welche vom Aloha Protokoll nicht verarbeitet, sondern nur weitergegeben werden sollen. Das Medium bzw. das Netzwerk über welches das Aloha Protokoll Frames empfangen oder versenden soll ist gemäß einer Bus-Topologie aufgebaut. Bei der Verwendung einer Bus-Topologie sind alle Teilnehmer über ein passives Übertragungsmedium miteinander verbunden. Jeder Teilnehmer kann mit jedem anderen Teilnehmer kommunizieren. Bei einer Bus-Topologie kann des Übertragungsmedium allerdings immer nur von einem Teilnehmer gleichzeitig verwendet werden, somit kann immer nur ein Frame gleichzeitig erfolgreich übertragen werden. Die Tatsache, dass immer nur ein Paket gleichzeitig übertragen werden kann, ist für das Aloha Protokoll ein nicht vernachlässigbarer Nachteil. Die Teilnehmer agieren bei der Verwendung des Protokolls unabhängig voneinander und prüfen vor Beginn einer Datenübertragung nicht die aktuelle Aktivität auf dem Übertragungsmedium. Das unabhängige Agieren voneinander hat zur Folge, dass eine Datenübertragung zu jedem beliebigen Zeitpunkt beginnen kann. Die Kombination von der beliebigen Wahl eines Startzeitpunktes für eine Übertragung und der fehlenden Überprüfung von bereits vorherrschender Aktivität und der Tatsache, das nur ein einzelner Frame gleichzeitig übertragen werden kann, führt dazu, dass nur in etwa 18.4\% der Zeit für erfolgreiche Übertragungen genutzt werden kann \cite{Back_AlohaPure}. In der restlichen Zeit treten Kollisionen auf, gesendete Daten werden nicht erfolgreich übertragen \cite{Back_AlohaPure}. Im Zusammenhang mit dem Aloha Protokoll bezeichnet eine Kollision einen fehlgeschlagenen Versuch einen Frame zu übertragen. Eine Kollision tritt also immer dann auf, wenn mehrere Teilnehmer gleichzeitig versuchen einen Frame zu übertragen. Diese Frames können nicht mehr voneinander unterschieden werden und sind deshalb für die anderen Teilnehmer nicht verständlich. Nach dem Auftreten einer Kollision wird ein zuvor kollidierter Frame allerdings nicht sofort wieder gesendet. Vor einem erneuten Versuch den Frame zu übertragen wartet der Teilnehmer eine gewisse Zeit, welche zufällig bestimmt wird. Durch diese Wartezeit versucht der Teilnehmer die Übertragungen der anderen Teilnehmer abzuwarten und so zu einem Zeitpunkt ohne Aktivität auf dem Übertragungsmedium den Frame erfolgreich übertragen.
\begin{figure}[h!]
	\begin{center}
	\includegraphics[scale=0.6]{img/ZeichnungExport2.png}
	\caption{Datenübertragung mit Kolisionen unter Verwendung des Aloha Protokolls}
	\end{center}
	\label{Abb2_PureAloha}
\end{figure}

. In den Kanälen A bis D ist jeweils die Aktivität eines einzelnen Teilnehmers enthalten. Jedes der abgebildeten Kästchen steht für eine Frame. An den auf verschiedenen Höhen der Zeitachse gelegenen Anfängen der Frames ist die Beliebigkeit des Starts einer Übertragung erkennbar. Die grau dargestellten Kästchen stehen für Frames, welche mit anderen Frames kollidiert sind. Jedes in grau dargestellte Kästchen überschneidet sich mit einem anderem in grau dargestellten Kästchen. Jedes weiße Kästchen überschneit sich mit keinem anderem Paket, ist folglich nicht kollidiert und wurde erfolgreich übertragen.\\
Eine Weiterentwicklung des Aloha Protokolls nimmt an einigen Stellen Verbesserungen vor. Diese Weiterentwicklung nennt sich Slotted Aloha \cite{Back_AlohaPure}. Die Zeit wird in feste Abschnitte eingeteilt, wobei ein Zeitabschnitt der Übertragungsdauer eines Frames entspricht. Der Beginn eines solchen Abschnittes sind auch die Zeitpunkte an denen mit der Übertragung begonnen werden kann. Durch die Festlegung von solchen Zeitpunkten wird eine Kollision schneller entdeckt und es werden weniger Daten übertragen, welche kollidieren. Wird zu beginn der Übertragung keine Kollision festgestellt, wird dies auch nicht am Ende festgestellt. Durch diese Verbesserung wurde der Anteil der Zeit, in der erfolgreich Daten übertragen werden auf etwa 36.8\% erhöht werden, ist also etwa doppelt so hoch wie beim herkömmlichen Aloha. \cite{Back_AlohaPure}.

\section{Elektrischer Strom}
Elektrischer Strom, Elektrizität oder umgangssprachlich auch Strom, all diese Begriffe bezeichnen die Bewegung geladener Teilchen entlang eines Leiters. Die geladen, sich bewegenden Teilchen sind gemäß geltenden physikalischen Gesetzen der Elektronenstromrichtung, in einem geschlossenen Stromkreis negativ geladen, es handelt sich also um Elektronen. Für eine erfolgreiche Übertragung von elektrischem Strom müssen genug dieser geladen Teilchen, mit der jeweils gleichen Ladung, vorhanden sein. Die Übertragung der geladenen Teilchen erfolgt über ein Medium welche genug dieser Ladungsträger verfügbar hat, ein solches Medium wird auch als Leiter bezeichnet.\\
Der elektrische Strom wird von vier Faktoren definiert, der Spannung U, der Stromstärke I, dem Widerstand R und der Leistung P. Die Spannung U, wird in der Einheit Volt (V) angegeben. Die Spannung gibt an welche Kraft auf die beweglichen Ladungsträger wirkt, je größer die Spannung, desto stärker bewegenden sich die Ladungsträger. Die Stromstärke wird in Ampere (A) angegeben und gibt an, wie viele Ladungsträger in einer Zeiteinheit durch einen Leiter fließen. Je höher die Stromstärke, desto mehr Ladungsträger fließen durch den Leiter. Der Widerstand angegeben in R, gibt an wie sehr die geladen Teilchen bei ihrem Fluss durch den Leiter gestört werden. Die Leistung P wird angegeben in Watt (W) und gibt an wie viel Energie/ Leistung übertragen wurde. Diese vier Faktoren sind untereinander so mit einander verbunden, dass mit der Formel 
\begin{align}
	U {=} R \cdot I
	\label{strom_formel_1}
\end{align}
, sowie ihren mathematischen Transformationen, die Spannung, die Stromstärke und der Widerstand in Verhältnis gesetzt werden können. Die elektrische Leistung P wird berechnet durch
\begin{align}
	P {=} U \cdot I
	\label{strom_formel_2}
\end{align}
, durch einsetzen von Formel \ref{strom_formel_1} in Formel \ref{strom_formel_2} kann P mit jeder Kombination von Spannung, Stromstärke und Widerstand bestimmt werden.\\
Die Spannung kann im Leiter auf verschiedene Arten vorliegen, in Form von Gleichspannung, Wechselspannung oder Mischspannung, wobei dies eine Kombination der ersten beiden darstellt. Gleichspannung fließt mit immer gleicher Stärke und immer gleicher Richtung durch den Leiter. Im Falle der Wechselspannung wechselt sowohl die Stärke, als auch die Flussrichtung in periodischen Abständen. Der Verlauf der Spannung während eines Wechselvorgangs kann verschiedene Formen annehmen, abgebildet auf Kurven kann ein rechteckiger, ein gezahnter, ein dreieckiger oder ein sinusförmiger Verlauf entstehen. Der im Stromnetz verwendet Wechsel entspricht einem sinusförmigen Verlauf. Der Widerstand hängt mit am stärksten vom verwendeten Leiter ab, je besser der Leiter geeignet ist, desto geringer ist der Widerstand. Der Widerstand eines Leiters ist auch von der Länge des Leiters abhängig, je Länger ein Leiter ist, desto größer ist sein Widerstand. Bei der elektrischen Leistung muss zwischen der Wirkleistung (Formel 2) und der Blindleistung unterschieden werden. Wirk- und Blindleistung bilden zusammen die Scheinleistung. Die Wirkleistung bezeichnet den Teil der elektrischen Leistung, welcher effektiv genutzt werden kann. Die Blindleistung bezeichnet den Teil welcher zwar ins Netz eingespeist werden muss aber nicht von seinen Nutzern verbraucht werden kann. Die Scheinleistung bezeichnet nun also die Summe von Wirk- und Blindleistung also alle Leistung, welche ins Netz eingespeißt wird.


\cite{widerstand_1} \cite{spannung_1} \cite{stromstaerke_1}

\section{Aufbau des Stromnetz}
Bei dem deutschen Stromnetz handelt es sich um ein Wechselspannungsnetz mit einer Normfrequenz von 50 Hz. Das Stromnetz lässt sich in zwei Ebenen einteilen, das Übertragungsnetz und das Verteilnetz. Das Übertragungsnetz ist ausgelegt auf die Übertragung von elektrischer Leistung über weite Strecken. Das Übertragungsnetz ist auch als Hochspannungsnetz bekannt. Dies rührt daher, dass die Spannung im Übertragungsnetz höher ist als im Verteilnetz. Die Spannung ist höher, da die Transportverluste bei höheren Wechselspannungen geringer ausfallen als bei niedrigeren. Je mehr Verluste bereits beim Transport auftreten, muss mehr Leistung bereitgestellt werden um dieselbe Leistung zum Abnehmer zu transportieren. Diese Abnehmer sind zu großenteilen mit dem Verteilnetz verbunden. Im Verteilnetz herrscht aber eine andere Spannung als im Verteilnetz. Diese verschiedenen Spannungen können mithilfe eines Transformators ineinander umgewandelt werden. Einzelne Verbraucher sind auch direkt ans Übertragungsnetz angeschlossen, aufgrund ihres hohen Leistungsbedarfs. Diese Großverbraucher verfügen über eigene Transformatoren.
\begin{figure}[h!]
	\includegraphics[width=\linewidth]{img/Stromnetz1.png}
	\caption{Aufbau des Stromnetzes}
	\label{Abb1_Stromnetz}
\end{figure}


Erläuterungen zum Bild. \\
Das Verteilnetz wird auch als Niederspannungsnetz bezeichnet, da es die niedrigste Spannung aller Stromnetze im deutschen Stromnetz aufweist. Ein Niederspannungsnetz verfügt im Normalfall über nur einen Transformator, welcher die elektrische Leistung für alle Abnehmer bereitstellt. Ein Niederspannungsnetz in Deutschland kann als kann als Strahlen-, Ring- oder Maschennetz betrieben werden. Ring- und Maschennetze bieten eine höhere Versorgungsicherheit, allerdings sind Strahlennetze kostengünstiger zu realisieren. Daher wird ab nun bei einem Niederspannungsnetz von einer Strahlentopologie ausgegangen. Bei einer Strahlentopologie teilt sich die Niederspannung Seite des Generators in mehrere Leitungen auf, wobei an jeder dieser Leitungen mehrere Teilnehmer verbunden sind. Ein solcher Strahl kann mit der Bustopologie des Aloha Protokolls verglichen werden. Jeder Teilnehmer eines solchen Strahls ist über ein geteiltes Medium mit jedem anderen Teilnehmer des Strahls verbunden. Lediglich der Transformator nimmt eine Sonderstellung ein, da er gemäß der Strahlentopologie, mit allen vorhandenen Strahlen verbunden ist, und so auch mit mehreren Bussen verbunden ist. 


\section{Grundlagen zu Elektrofahrzeugen}
Ein Fahrzeug kann dann als Elektrofahrzeug bezeichnet werden, wenn es in der Lage ist elektrische Energie für seine unmittelbare Fortbewegung zu nutzen. Dieses Nutzen kann auf mehrere Arten erreicht werden. Als erste zu nennen ist das batterieelektrische Fahrzeug oder Battery Electric Vehicle (BEV), bei dieser Art der Bauweise dient ein Akku als Energiespeicher für einen oder mehr Elektromotoren. Die Energie wird in einem Akku gespeichert. Dieser Akku wird entweder durch Rekuperation, also durch Rückumwandlung von Fortbewegungsenergie in elektrische Energie, oder durch einen Ladevorgang an einem Ladegerät geladen. Eine andere Art eines Batterie-elektrischen Fahrzeuges wäre, ein batterie-elektrisches Fahrzeug mit Range-Extender. Bei diesen kann der Akku auch mithilfe eines, im Fahrzeug verbauten, Verbrennungsmotors geladen werden. Dieser Verbrennungsmotor treibt einen Generator an, wodurch elektrischer Strom erzeugt wird, welcher dann in der Batterie gespeichert werden kann. Der Verbrennungsmotor ist aber nicht in der Lage das Fahrzeug direkt anzutreiben, wie bei der klassischen Verwendung des Verbrenners in einem Fahrzeug. Der Verbrennungsmotor ist des Weiteren nicht in der Lage die volle Leistung der verbauten Elektromotoren zu bedienen. Bei leerem Akku ist die Leistung des Fahrzeuges limitiert durch die Leistung des Verbrenners. \\
Neben dem Konzept des batterie-elektrischen Antriebes gibt auch Hybride Ansätze, wo die Leistung eines Verbrenners und einem oder mehrere Elektromotoren kombiniert wird. Diese Ansätze lassen in drei Gruppen unterteilen. Bei der ersten Gruppe, generiert ein Verbrenner, mit einem Generator, oder eine Brennstoffzelle, die elektrische Energie, welche der Elektromotor für den Antrieb benötigt. Der verbaute Akku dient nur zum speichern für kurze Zeit und kann nicht von außen geladen werden. In der zweiten Gruppe, dienen die oder der verbaute Elektromotor nur zur Unterstützung des Verbrenners, nicht allerdings zum alleinigen Antrieb des Fahrzeugs. Die verbaute Batterie hat hier ebenfalls keine hohe Kapazität und kann nicht von außen geladen werden. Die dritte Gruppe beinhalten nun Systeme, welche als Hybrid Electric Vehicle (HEV) oder Plug-In Hybrid Electric Vehicle (PHEV) bekannt sind. In beiden Fahrzeugtypen ist ein Verbrennungsmotor und einer oder mehrere Elektromotoren verbaut. Anders als bisher sind hier aber beide Motorarten jeweils alleine in der Lage das Fahrzeug zu betreiben. Fahrzeuge dieser Kategorie können wie ein herkömmlicher Verbrenner oder wie ein BEV verwendet werden. Das Merkmal was ein PHEV von einem PEV unterscheidet ist, dass bei einem PHEV die verbaute Batterie von außen über ein Ladegerät geladen werden kann. Die Ladung über einen, vom Verbrennungsmotor angetriebenen, Generator oder über Rekuperation ist hingegen sowohl beim HEV als auch beim PHEV möglich.\\
Innerhalb dieser Arbeit werden jene Arten von Elektrofahrzeugen betrachtet, deren Batterie mithilfe eines Ladegerätes zu laden. Zu dieser Art zählen die Battery Electric Vehicles (BEV), BEV mit Range-Extender und Plug-In-Hybrid Electric Vehicles. Alle anderen vorgestellten Elektrofahrzeuge können die verbaute Batterie nicht über ein externes Ladegerät laden. Bei den verfügbaren Ladegeräten gibt es verschiedene Techniken. Die erste Unterscheidung liegt beim Gleichstrom- und Wechselstromladen. Ein Akku, wie er in einem Elektrofahrzeug verbaut ist, speichert die Energie in Form von Gleichstrom, nicht in Form von Wechselstrom. Entfällt die Umwandlung von Wechsel- auf Gleichstrom, kann die maximal mögliche Ladeleistung gesteigert werden. Beim gleichstromladen sind aktuell etwa 200kW Ladeleistung möglich, Ladestation gibt es aber meist nur im öffentlichen Raum. Ladegeräte mit Wechselspannungstechnik gibt es auch für den privaten Bereich. Beim Wechselspannungsladen wird die Wechselspannung erst in einem internen Ladegerät zur, für die Batterie passenden, Gleichspannung. Dieses interne Ladegerät gibt meist auch die maximale Ladeleistung des Fahrzeuges vor. Die Leistung des externen Ladegeräts variiert ja nach Technik und Anschluss ans Stromnetz. Die mögliche Ladeleistung des externen Ladegeräts reicht von 3,7kW, 11kW bis zu 22kW. Wobei die maximale Leistungsfähigkeit des Stromanschlusses zu beachten ist, ein 3,7kW und 11kW Ladegerät sind in Normfällen immer möglich, während für ein 22kW Ladegerät unter Umständen der Anschluss ans Stromnetz modifiziert werden muss, bevor es mit voller Leistung betrieben werden kann.


